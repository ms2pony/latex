% !Mode:: "TeX:UTF-8"

\documentclass[final]{jcr}
%选项 preprint 提供手稿预览, 交叉引用颜色为蓝色
%    final    提交手稿时使用, 所有交叉引用颜色改为黑色
%    review   匿名评审模式, 打开之后编译时隐去作者信息

\usepackage[ruled]{algorithm2e}
\usepackage{hyperref}
\usepackage{bm}
\usepackage{lastpage}
\usepackage{enumitem}

\begin{document}

%% 编辑输入区 %%
\DOI{10.13868/j.cnki.jcr.000XXX}%文章DOI号
\vol{7}% 卷号
\issue{1}% 期号
\PrintDate{2020}{10}% 出版日期, 前者为年, 后者为月
\ReceivedDate{2019-05-01}% 收稿日期
\FinalVersionDate{2019-09-01}% 定稿日期
\setcounter{page}{1}
\PrintPage{\thepage}{\pageref{LastPage}}% 论文起止页码
\AuthorCiteInput{ZUO Z Y, ZUO Z E, ZUO Z S}%输入英文引用格式中的英文作者姓名
%% 编辑输入区 %%

%% 以下内容均由作者提供 %%
\DocumentCode{A}%输入文献标识码
\CLCnumber{TP309.7}%输入中图分类号

\begin{frontmatter}
  % 输入中文标题
  \title{AES算法研究}

  % 输入英文标题
  \etitle{Research on \uppercase{AES}}

  % 输入基金项目
  \foundation{国家重点研发计划(XXXX); 国家自然科学基金 (XXXX)}
  \efoundation{National Key Research and Development Program of China (XXXX); National Natural Science Foundation of China (XXXX)}
  %------------- 输入作者信息 --------------%
    % 如果所有作者都在同一单位, 输入方法如下:
    % \author{作者一}% 输入作者中文姓名
    % \author{作者二}% \author 命令可以重复使用
    % \ead{Email}% 输入通讯作者的电子邮箱, 须紧随\author命令输入
    % \cortext{通讯作者}% 输入通讯作者, 此时通讯作者为上一个\author命令输入的作者, 并结合\ead命令显示通讯作者邮箱
    % \address{作者单位}% 输入作者单位

    % 如果多个作者在不同单位, 输入方法如下:
    % \author[biaoqian1,biaoqian2]{作者一}% 方括号内是引用自\address的标签
    % \address[biaoqian1]{作者一的第一单位} % 方括号内是标签, 由\author命令引用
    % \author[biaoqian2]{作者二}% 输入第二作者中文姓名
    % \ead{Email}% 输入通讯作者的电子邮箱, 须紧随\author命令输入
    % \cortext{通讯作者}% 输入通讯作者, 此时通讯作者为上一个\author命令输入的作者, 并结合\ead命令显示通讯作者邮箱
    % \address[biaoqian2]{作者一的第二单位}% 输入作者单位

    % 作者英文信息与上述方法相同
  %------------- 输入作者信息 --------------%

  % 输入作者中文及通讯作者信息
  \author[1]{作者一}
  \ead{E-mail: zuozhe1@net.cn}
  \cortext{通信作者}
  \author[1,2]{作者二}
  \author[1,2]{作者三}

  % 输入作者英文及通讯作者信息
  \eauthor[1]{ZUO Zhe-Yi}
  \ecortext{Corresponding author}
  \eauthor[1,2]{ZUO Zhe-Er}
  \eauthor[1,2]{ZUO Zhe-San}

  % 输入作者附属单位中文
  \address[1]{XX大学 XXXX实验室, 济南\ 250100}
  \address[2]{XXX研究院, 北京\ 100190}

  % 输入作者附属单位英文
  \eaddress[1]{Lab of XXXX, XXXX University, Jinan 250100, China}
  \eaddress[2]{Academy of XXXX, Beijing 100190, China}

  % 输入中文摘要
  \begin{abstract}
    本刊要求来稿摘要内容详实, 字数不少于400字, 能全面表述稿件的主要观点和结论, 便于读者通过阅读摘要了解到作者的研究内容、方法和主要成果, 同时要求英文摘要对照准确.
  \end{abstract}

  % 输入中文关键词
  \begin{keywords}
    AES算法; 差分攻击
  \end{keywords}

  % 输入英文摘要
  \begin{eabstract}
    XXX.
  \end{eabstract}

  % 输入英文关键词
  \begin{ekeywords}
    AES; differential cryptanalysis
  \end{ekeywords}
\end{frontmatter}

\zihao{-5}
\section{一级标题1}

正文部分\upcite{1}.文献\cite{1}说了什么

\section{一级标题2}
\subsection{二级标题1}
\subsubsection{三级标题1}

\begin{theorem}
	定理内容.
\end{theorem}

\begin{proof}
	证明过程.
\end{proof}

\begin{definition}
	定义内容.
\end{definition}

\begin{lemma}
	引理内容.
\end{lemma}

\begin{corollary}
	推论内容.
\end{corollary}

\begin{proposition}
	命题内容.
\end{proposition}

\begin{remark}
	备注内容.
\end{remark}

\begin{example}
	例子内容.
\end{example}

\begin{assumption}
	假设条件.
\end{assumption}

\begin{algorithmn}
	算法内容.
\end{algorithmn}

\renewcommand{\algorithmcfname}{\normalfont\kaishu  \zihao{-5} 算法}
\SetAlgoCaptionSeparator{~}
\begin{algorithm}[H]
	\zihao{6}
	\caption{\kaishu \zihao{-5} AES算法}%算法名字
    \label{alg1}
	\LinesNumbered %要求显示行号
	\KwIn{input parameters A, B, C}%输入参数
	\KwOut{output result}%输出

	\For{$\mathrm{condition}$}{
		only if\;
		\If{$\mathrm{condition}$}{
			1\;
		}
	}
\end{algorithm}

\begin{align}
	{\deg}(g_{y_1})\leq \min⁡\{&{\rm DEG}(y_{t-r_C-r_B-1})+{\rm DEG}(z_{t-r_C}),\\                              
	&{\rm DEG}(y_{t-r_C-r_B+1})+{\rm DEG}(z_{t-r_C-1}),\\
	&{\rm DEG}(y_{t-r_C-r_B-1})+{\rm DEG}(y_{t-r_C-r_B})+{\rm DEG}(y_{t-r_C-r_B+1})\}\\
	&\stackrel{\Delta}{=}d1.\\
\end{align}
\section{一级标题3}

公式示例. 结论性公式没有必要全部编号, 后面要引用才需编号.
\begin{equation}\label{eq1}
a^2+b^2=c^2
\end{equation}

\begin{thebibliography}{99}
	\bibitem{1} DE CANNI{\` E}RE C, DUNKELMAN O, KNE{\v Z}EVI{\'C} M. KATAN and KTANTAN---a family of small and efficient hardware-oriented block ciphers[C]. In: Cryptographic Hardware and Embedded Systems—CHES 2009. Springer Berlin Heidelberg, 2009: 272--288. [DOI: 10.1007/978-3-642-04138-9\_20]
	\bibitem{2} LI Q, FENG D G, ZHANG L W, et al. Enhanced attribute-based authenticated key agreement protocol in the standard model[J]. Chinese Journal of Computers, 2013, 36(10): 2156–2167. [DOI: 10.3724/SP.J.1016.2013.02156]\\
李强, 冯登国, 张立武, 等. 标准模型下增强的基于属性的认证密钥协商协议[J]. 计算机学报, 2013, 36(10): 2156--2167. [DOI: 10.3724/SP.J.1016.2013.02156]
\end{thebibliography}
\newpage
\authorinfo
\begin{AuthorInfo}
  \begin{jcrbiography}[{\includegraphics[width=1.05in]{fig/zuozhe}}]{作者一} (1989--), 河南郑州人, 博士生在读. 主要研究领域为对称, 密码算法的安全性分析.\\ zuozhe1@net.cn
  \end{jcrbiography}
  \begin{jcrbiography}[{\includegraphics[width=1.05in]{fig/zuozhe}}]{作者二} (1982--), 山东济南人, 教授. 主要研究领域为对称, 密码算法的安全性分析. \\  zuozhe2@net.cn
  \end{jcrbiography}
\end{AuthorInfo}
\begin{AuthorInfo}
  \begin{jcrbiography}[{\includegraphics[width=1.05in]{fig/zuozhe}}]{作者三} (1989--), 北京人, 副研究员. 主要研究领域为对称, 密码算法的安全性分析.\\ zuozhe3@net.cn
  \end{jcrbiography}

\end{AuthorInfo}
\vspace{10em}
{\noindent\kaishu 附: 图、表模板}

\begin{table}
	\centering
	\caption{三线式表格}{Three lines table}
	\label{tab1}
	\tabulinesep=1.5mm
	\begin{tabu}to \linewidth{X[c,m]X[c,m]X[c,m]X[c,m]}
		\tabucline[0.08em]-
		\textbf{列1} & \textbf{列2 }& \textbf{列3} & \textbf{列4}\\
		\tabucline-
		行1 & XX & XX & XX \\
		行2 & XX & XX & XX \\
		行3 & XX & XX & XX \\
		行4 & XX & XX & XX \\
		\tabucline[0.08em]-
	\end{tabu}
\end{table}

\vspace{2ex}
\begin{figure}
	\centering
	\includegraphics[width=0.3\linewidth]{fig/zuozhe}
	\caption{示例}{Example}
	\label{fig1}
\end{figure}

\end{document}
